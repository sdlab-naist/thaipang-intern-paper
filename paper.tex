\documentclass[conference]{IEEEtran}
\usepackage{microtype,mathtools,amsmath,multibib,amsfonts,multicol,array,multirow,place ins,cite,makecell,algorithm,algorithmic,subfigure,paralist,graphicx}
%\usepackage{flushend}

\usepackage[font=small,labelfont=bf]{caption}

\usepackage{xspace}
\usepackage{color}
\usepackage{ifthen}
\usepackage{url}
\usepackage{fancybox}


\graphicspath{{figures/}}
\DeclareGraphicsExtensions{.pdf,.jpeg,.png}
\hyphenation{op-tical net-works semi-conduc-tor}


\begin{document}
\title{Title}

\maketitle
\newboolean{showcomments}
\setboolean{showcomments}{true} % toggle to show or hide comments
\ifthenelse{\boolean{showcomments}}
{\newcommand{\nbnote}[2]{
  % \fbox{\bfseries\sffamily\scriptsize#1}
  \fcolorbox{blue}{yellow}{\bfseries\sffamily\scriptsize#1}
  {\sf\small\textit{#2}}
  % \marginpar{\fbox{\bfseries\sffamily#1}}
 }
}
{\newcommand{\nbnote}[2]{}
 \newcommand{\version}{}
}
\newcommand\pick[1]{\nbnote{Pick sez}{\textcolor{magenta}{#1}}}
\newcommand\thai[1]{\nbnote{Thai sez}{\textcolor{blue}{#1}}}


\begin{abstract}

% the freeform style of modern code review

% some comments are not technically contributing


\begin{IEEEkeywords}
Peer Code Review, Software Inspection, Reviewer Recommendation, Reviewer Assignment
\end{IEEEkeywords}
\end{abstract}

\section{Introduction}
Motivation Example.\thai{Wow}
\begin{description}
\item[RQ1:] Can we identify useful and useless discussions in code review?
\item[RQ2:] How impact of useless discussion impact to the software quality?
\end{description}

\section{Background}
\subsection{Modern Code Review}
\subsection{Text Mining Techniques}
\subsubsection{tf--idf}
\subsubsection{Similarity Measure}
\subsection{Evaluation Techniques}
\subsubsection{Precision and Recall}
\subsubsection{F-measure}


\section{Classification Method}

Blah.


\subsection{Data Preparation}

% where we got the data
We used the review data sets of the Qt project collected by Hamasaki et al.
Only reviews in the master branch of \texttt{qtbase} project are considered,
as it is the most active branch.

% quick structure
Each change includes a \emph{commit message}, which describes what is changed.
After a change is submitted to Gerrit, reviewers can give scores to it, which will decide if the change will be accepted or not.
Reviewers can also add comments to the change.
Comments may optionally be added to a specific line of code inside a changed file.

% conversion into vector, wow
The commit messages and all comment texts are converted into vectors as follows.
Firstly, we extracted words from the text, removed the stop words, and put the remaining words through Porter stemming algorithm.
Secondly, we combined all the remaining tokens to form a corpus of all words from all documents.
Finally, we used tf--idf to convert each document into a vector.


\subsection{Training Data}

We sampled 320 comments from the historical data of code reviews in the Qt project.
We then classified the samples as either useful (that is, technically contributing to the code) or not useful.
This task is carried out by three people who worked independently.


%�� Overview (No need for subsection)
%\subsection{Classifying useful discussion using text similarity and dissimilarity}
%Need a better section name - -.
%\subsection{Validation}
% (10-fold cross validation)
\section{Results}
\subsection{Studied Project and Data set}
\subsection{RQ1: Can we identify useful and useless discussion in code review?}
\subsection{RQ2: How impact of useless discussion impact to the software quality?}
\section{Threat to Validity}
\section{Conclusion and Future Work}


\IEEEpeerreviewmaketitle

\bibliographystyle{IEEEtran}

%\bibliography{references}



% that's all folks
\end{document}



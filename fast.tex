\documentclass[conference]{IEEEtran}
\usepackage{microtype,mathtools,amsmath,multibib,amsfonts,multicol,array,multirow,place ins,cite,makecell,algorithm,algorithmic,subfigure,paralist,graphicx}
%\usepackage{flushend}

\usepackage[font=small,labelfont=bf]{caption}

\usepackage{xspace}
\usepackage{color}
\usepackage{ifthen}
\usepackage{url}
\usepackage{fancybox}


\graphicspath{{figures/}}
\DeclareGraphicsExtensions{.pdf,.jpeg,.png}
\hyphenation{op-tical net-works semi-conduc-tor}


\begin{document}
\title{Title..?}

\maketitle
\newboolean{showcomments}
\setboolean{showcomments}{true} % toggle to show or hide comments
\ifthenelse{\boolean{showcomments}}
{\newcommand{\nbnote}[2]{
  % \fbox{\bfseries\sffamily\scriptsize#1}
  \fcolorbox{blue}{yellow}{\bfseries\sffamily\scriptsize#1}
  {\sf\small\textit{#2}}
  % \marginpar{\fbox{\bfseries\sffamily#1}}
 }
}
{\newcommand{\nbnote}[2]{}
 \newcommand{\version}{}
}
\newcommand\pick[1]{\nbnote{Pick sez}{\textcolor{magenta}{#1}}}
\newcommand\thai[1]{\nbnote{Thai sez}{\textcolor{blue}{#1}}}


\begin{IEEEkeywords}
Code review, Software inspection, Text mining
\end{IEEEkeywords}



\section{Introduction}

Modern, lightweight code reviews are widely-used due to its low process overhead,
when compared to formal software inspections, which involves many participants and tedious processes.

In prior works, it has been shown
that the coverage of formal \thai{cite from the citation found in McIntosh's paper}
and informal \thai{cite McIntosh} code review tend to affect the software quality.
However, the free-form nature of informal code reviews makes it harder to evaluate the review's quality and coverage.

In this paper,
we try to find out whether the similarity of code review text correlates with the resulting software quality,
by analyzing the historical reviews and defects records of open-source projects.


\section{Approach}

We sampled 320 comments from the historical data of code reviews in the Qt project.
We then classified the samples as either useful (that is, technically contributing to the code) or not useful.
This task is carried out by three people who worked independently.
We then discarded the samples where people don't agree on its usefulness.

The comment texts,
along with the description of the change that is being reviewed,
are pre-processed by removing stop words and by stemming,
and then converted into pairs of vectors using tf--idf algorithm.
These pairs of vectors are then compared using cosine similarity and euclidean distance measures.

The threshold value for these measures can be found
by trying every possible combination
and searching for a set of value that gives the highest score (we used $F_1$ score).
This is possible because there is a finite number of values in each measure.











\IEEEpeerreviewmaketitle

\bibliographystyle{IEEEtran}

%\bibliography{references}



% that's all folks
\end{document}



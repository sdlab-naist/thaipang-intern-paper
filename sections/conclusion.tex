% !TEX root = ../paper.tex

\section{Conclusions}

In this paper, we propose a practical approach to classify usefulness discussion in MCR based on VSM similarity.
The results from our case study show that our approach can classify comments with better performance than a random model both for precision and recall.
In addition, we found that 85\% of comments in Gerrit are automatically generated and not likely useful (but important for record keeping).
For the remaining 15\%, about 43\% of them are automatically classified as useful, while about 34\% are classified as not useful.
The remaining 23\% could not be classified by model, and required manual classification.
Thus the classification effort is reduced by about 77\%.

More work could be done in the future to relate the amount of useful comments to the software quality.
This could enable a way of assessing process quality in a quantitative way.
Future work could also consider different models other than semantic similarity.